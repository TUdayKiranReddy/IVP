% CVPR 2022 Paper Template
% based on the CVPR template provided by Ming-Ming Cheng (https://github.com/MCG-NKU/CVPR_Template)
% modified and extended by Stefan Roth (stefan.roth@NOSPAMtu-darmstadt.de)

\documentclass[10pt,twocolumn,letterpaper]{article}

%%%%%%%%% PAPER TYPE  - PLEASE UPDATE FOR FINAL VERSION
%\usepackage[review]{cvpr}      % To produce the REVIEW version
\usepackage{cvpr}              % To produce the CAMERA-READY version
%\usepackage[pagenumbers]{cvpr} % To force page numbers, e.g. for an arXiv version

% Include other packages here, before hyperref.
\usepackage{graphicx}
\usepackage{amsmath}
\usepackage{amssymb}
\usepackage{booktabs}


% It is strongly recommended to use hyperref, especially for the review version.
% hyperref with option pagebackref eases the reviewers' job.
% Please disable hyperref *only* if you encounter grave issues, e.g. with the
% file validation for the camera-ready version.
%
% If you comment hyperref and then uncomment it, you should delete
% ReviewTempalte.aux before re-running LaTeX.
% (Or just hit 'q' on the first LaTeX run, let it finish, and you
%  should be clear).
\usepackage[pagebackref,breaklinks,colorlinks]{hyperref}


% Support for easy cross-referencing
\usepackage[capitalize]{cleveref}
\crefname{section}{Sec.}{Secs.}
\Crefname{section}{Section}{Sections}
\Crefname{table}{Table}{Tables}
\crefname{table}{Tab.}{Tabs.}


%%%%%%%%% PAPER ID  - PLEASE UPDATE
\def\cvprPaperID{*****} % *** Enter the CVPR Paper ID here
\def\confName{CVPR}
\def\confYear{2022}


\begin{document}

%%%%%%%%% TITLE - PLEASE UPDATE
\title{Team 4 - Preliminary Project Report}

\author{Tadipatri Uday Kiran Reddy\\
{\tt\small ee19btech11038@iith.ac.in}
% For a paper whose authors are all at the same institution,
% omit the following lines up until the closing ``}''.
% Additional authors and addresses can be added with ``\and'',
% just like the second author.
% To save space, use either the email address or home page, not both
\and
Sahukari Chaitanya Varun\\
{\tt\small ee19btech11040@iith.ac.in}
\and
L. Pranay Kumar Reddy\\
{\tt\small ai19btech11010@iith.ac.in}
}
\maketitle

%%%%%%%%% ABSTRACT
\begin{abstract}
With the evolution of high resolution digital cameras, snapshots have become a most common way to record and share visual data and experiences taken through mobile phones and tablets. But as humans, it is difficult to keep a stiff hold while capturing image, especially, when image is to be captured on move introducing blur. Though there lot of external tools for frame stabilization such as gimbal, mounting it on can be difficult at times. Hence, we propose this paper to solve this problem by processing the image with the aid of gyroscope sensed data which is attached with the camera device, helping to deblur the image. We aim to implement this deblurring with the help of deep convolution nets and extrapolate it with GAN's to give better quality and faster output than compared to traditional CNN's. With faster computation capabilities, we can extend this work to stabilise the video feed with lower frame rates.
\end{abstract}

%%%%%%%%% BODY TEXT
\section{Introduction}
\label{sec:intro}
%Introduce the problem you are working on. This section should give a general introduction motivation for the problem. You can also describe the applications of your problem here. 
%
%Please number all of your sections and displayed equations as in these examples:
%\begin{equation}
%  E = m\cdot c^2
%  \label{eq:important}
%\end{equation}
%and
%\begin{equation}
%  v = a\cdot t.
%  \label{eq:also-important}
%\end{equation}
Most of the times, our desired scenes are in always in motion, capturing this in camerea would introduce a fuzzy noise which is known as motion blur. This affect is also very severe in Autonoumus vehichle which utilise sophosticated visual SLAM algorithms on live feed coming from camera or LIDAR sensors which are in motion. Here the motion blur would supress the important features needed to be obtained from the video feed. Even with state-of-the-art technology of camera manufacturing, motion blur affect is not being improved while the resolution of the cameras are getting better and better.
Mechanically stabilising the sensor like camera or LIDAR is very hard, so we have to use rigourus algorithms to process the image and many \textbf{recent works} like {} came up with both linear and non-linear models to deblur the image but the processed images are having still some blur and implementation of such sophosticated algorithm in edge devices like mobile phones, DSLR cameras, etc is computationally infeasible.
%------------------------------------------------------------------------
\section{Problem Statement}
\label{sec:problem}


\section{Literature Review}
\label{sec:literature}
This section should clearly and in detail describe the literature you have read, and its relevance to the problem being solved. You are 

\section{Preliminary Results}
\label{sec:prelim_results}
This is an optional section. This could include any results you have been able to replicate from the literature as well as any preliminary results that you want to report.

%-------------------------------------------------------------------------
\section{References}
List all the references you have read.

%%%%%%%%% REFERENCES
{\small
\bibliographystyle{ieee_fullname}
\bibliography{egbib}
}

\end{document}
